\documentclass[a4paper,12pt]{article}

\usepackage[utf8x]{inputenc}
\usepackage[T1]{fontenc}
\usepackage{lmodern}
\usepackage[ngerman]{babel}



%opening
\title{Leitfragen Woche 1}
\author{Floris Remmert\\\small{Mat. Nr. 4143457}}

\begin{document}

\maketitle

%\begin{abstract}
%
%\end{abstract}

\section{Erkl"arung von Leitbegriffen}
	\subsection{Begriffliches Wissen, Wissensatome (\textit{taṣawwur})}
		Dieser Begriff beschreibt das notwendig existierende, jene Dinge deren Wahrheit bereits in ihrer Existenz begründet ist. Dieses Wissen kommt ohne jegliche Form von Syllogismen zustande.
	\subsection{Propositionales Wissen, Wissensmoleküle (\textit{taṣdīq})}
		Dieses Wissen ergibt sich durch Syllogismen. Grundlage dafür sind als wahr akzeptierte Prämissen. Dieses Wissen bezeichnet Avicenna auch als das möglich existierende. Es ist die unter Syllogismen abgeschlossene Menge allen begrifflichen und propositionalen Wissens.
	\subsection{Prinzipien der Begriffsbildung}
		Jeder Begriff muss eine bestimmte Bedeutung haben, denn hat er es nicht, so l"asst er sich unm"oglich von etwas anderem abtrennen (vgl. \ref{a3}), sodass die Begriffsbildung unm"oglich wird.
	\subsection{Prinzipien der Beweisbildung}
		Avicenna beschreibt Syllogismen als Prinzipien der Beweisbildung. Diese sind entweder "`in sich selbst"', also derart dass sie zwei allgemein bekannte Schlussweisen unter korrekter Schlussweise zu einem weniger bekannten Schluss bringen, oder aber basieren auf Pr"amissen die nur die Beteiligten als wahr anerkennen. Entscheidend ist die korrekte Schlussweise.
	\subsection{Kontradiktorische Begriffe/Aussagen}
		Der Satz des ausgeschlossenen Dritten (\ref{a3}) erm"oglicht erst die Einf"uhrung kontradiktorischer Begriffe.
	\subsection{Satz vom ausgeschlossenen Dritten}\label{a3}
		F"ur jede Aussage gilt, dass sie entweder wahr, oder falsch ist. Angenommen dies w"are nicht so, so verl"oren wir die F"ahigkeit zwischen verschiedenen Begriffen zu differenzieren, sodass ein Dialog vollst"andig unm"oglich w"urde.
	
\section{Beantwortung der Leitfragen}
Wie und auf welchen Betrachtungsebenen versteht Avicenna „das Wahre“?
\begin{itemize}
	\item Wahrheit als dauerhafte Existenz (notwendig Existierendes)
	\item Wahrheit als Eigenschaft einer m"undlichen Aussage (m"oglich Existierendes)
	\item falsch
\end{itemize}
Wie stuft sich das Wahre auf der propositionalen Ebene ab?
\begin{enumerate}
	\item dauerhaft \& prim"ar: das notwendig Existierende, Dinge die aus sich heraus direkt wahr sind. Die "`wahrste"' Aussage ist der Satz des ausgeschlossenen Dritten.
	\item dauerhaft: Dinge die durch Syllogismen aus ersteren folgen
	\item alles Andere, insbesondere das m"oglich Existierende
\end{enumerate}
Wie weist Avicenna den aufrichtig verwirrten Sophist darauf hin, dass es zwischen zwei Kontradiktionen kein Mittleres gibt?
\begin{itemize}
	\item Angenommen es g"abe ein Mittleres
	\item jeder Begriff muss klar abgegrenzt sein. Bspw. ein Auto ist ein Auto, aber nicht ein Zug
	\item g"abe es ein mittleres, so w"are ein Auto auch ein Zug, und auch alles andere, gleichzeitig jedoch nicht alles andere.
	\item da das absurd ist, folgern wir, dass es kein Mittleres gibt
\end{itemize}

\section{Quiz}
\emph{In welchem Sinne ist ein relativ-Syllogismus ein Syllogismus und in welchem Sinne relativ?}

Er ist ein \emph{Syllogismus}, da es eine sichere Konklusion gibt, insofern den Pr"amissen zugestimmt wird. Er ist \emph{relativ}, da den Pr"amissen zugestimmt werden muss. Diese unterliegen einem gewissen Kontext, sodass man ihnen im Allgemeinen nicht unbedingt zustimmen mag.
\end{document}
