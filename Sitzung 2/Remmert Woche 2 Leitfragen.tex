\documentclass[a4paper,12pt]{article}

\usepackage[utf8x]{inputenc}
\usepackage[T1]{fontenc}
\usepackage{lmodern}
\usepackage[ngerman]{babel}

\usepackage{physics}
\usepackage{amsmath}
\usepackage{tikz}
\usepackage{mathdots}
\usepackage{yhmath}
\usepackage{cancel}
\usepackage{color}
\usepackage{siunitx}
\usepackage{array}
\usepackage{multirow}
\usepackage{amssymb}
\usepackage{gensymb}
\usepackage{tabularx}
\usepackage{booktabs}
\usetikzlibrary{fadings}
\usetikzlibrary{patterns}
\usetikzlibrary{shadows.blur}
\usetikzlibrary{shapes}

%opening
\title{Leitfragen Woche 2}
\author{Floris Remmert\\\small{Mat. Nr. 4143457}}

\begin{document}

\maketitle

%\begin{abstract}
%
%\end{abstract}

\section{Erkl"arung von Leitbegriffen}
	\subsection{Die Substanz und das Akzidens}
		Sie \emph{Substanz} ist das, was kein Subjekt ist, also nicht Teil einer Spezies ist, oder vergleichbar. Akzidens betrifft immer ein Subjekt. Sie ist rekursiv definiert und kann immer auf eine Substanz zur"uckgef"uhrt werden: Ist das Subjekt der Akzidens eine Substanz, so besteht die Akzidens in dieser Substanz, ist es selbst ein Subjekt, so beginne von vorn. Sp"ater zeigt Avicenna, dass diese Rekursion immer abbricht und deshalb jede Akzidens eine Substanz betrifft.
	\subsection{Das Aufnehmende und das Subjekt}
		Das Aufnehmende ist etwas, was erst und nur durch die Teilhabe anderer Subjekte entsteht. Das Subjekt besteht also aus sich selbst heraus, w"ahrend das Aufnehmende nur durch andere Subjekte bestehen kann.
	\subsection{Hylemorphismus}
		Hylemorphismus ist die aristotelische Theorie der Aufteilung des Seins in Materie und Form, so wie Avicenna sie hier auch vornimmt.
	\subsection{Unterteilungen der Substanz}
		Siehe Abbildung \ref{fig:1}.
		\begin{figure}[htb]
			
			
			
			
			
			\tikzset{every picture/.style={line width=0.75pt}} %set default line width to 0.75pt        
			
			\begin{tikzpicture}[x=0.75pt,y=0.75pt,yscale=-1,xscale=1]
			%uncomment if require: \path (0,422); %set diagram left start at 0, and has height of 422
			
			
			% Text Node
			\draw (26,6) node [anchor=north west][inner sep=0.75pt]   [align=left] {{\large \textbf{Substanz}}};
			% Text Node
			\draw (43.5,75) node [anchor=north west][inner sep=0.75pt]   [align=left] {Körper};
			% Text Node
			\draw (199.5,75) node [anchor=north west][inner sep=0.75pt]   [align=left] {kein Körper};
			% Text Node
			\draw (115,137.5) node [anchor=north west][inner sep=0.75pt]   [align=left] {Teil eines Körpers};
			% Text Node
			\draw (293,127) node [anchor=north west][inner sep=0.75pt]   [align=left] {\begin{minipage}[lt]{97.48956pt}\setlength\topsep{0pt}
				\begin{center}
				komplett unabhängig\\von Körpern
				\end{center}
				
				\end{minipage}};
			% Text Node
			\draw (93.5,250) node [anchor=north west][inner sep=0.75pt]   [align=left] {Form};
			% Text Node
			\draw (213,250) node [anchor=north west][inner sep=0.75pt]   [align=left] {Materie};
			% Text Node
			\draw (309.5,239.5) node [anchor=north west][inner sep=0.75pt]   [align=left] {\begin{minipage}[lt]{79.16356pt}\setlength\topsep{0pt}
				\begin{center}
				"`bewegt"' Körper:\\Seele
				\end{center}
				
				\end{minipage}};
			% Text Node
			\draw (480,229) node [anchor=north west][inner sep=0.75pt]   [align=left] {\begin{minipage}[lt]{58.298644pt}\setlength\topsep{0pt}
				\begin{center}
				frei von\\Materiellem:\\Intellekt
				\end{center}
				
				\end{minipage}};
			% Connection
			\draw    (69,34) -- (69,72) ;
			\draw [shift={(69,74)}, rotate = 270] [color={rgb, 255:red, 0; green, 0; blue, 0 }  ][line width=0.75]    (10.93,-3.29) .. controls (6.95,-1.4) and (3.31,-0.3) .. (0,0) .. controls (3.31,0.3) and (6.95,1.4) .. (10.93,3.29)   ;
			% Connection
			\draw    (106.22,34) -- (207.05,73.27) ;
			\draw [shift={(208.91,74)}, rotate = 201.28] [color={rgb, 255:red, 0; green, 0; blue, 0 }  ][line width=0.75]    (10.93,-3.29) .. controls (6.95,-1.4) and (3.31,-0.3) .. (0,0) .. controls (3.31,0.3) and (6.95,1.4) .. (10.93,3.29)   ;
			% Connection
			\draw    (228.3,99) -- (191.63,135.1) ;
			\draw [shift={(190.2,136.5)}, rotate = 315.45] [color={rgb, 255:red, 0; green, 0; blue, 0 }  ][line width=0.75]    (10.93,-3.29) .. controls (6.95,-1.4) and (3.31,-0.3) .. (0,0) .. controls (3.31,0.3) and (6.95,1.4) .. (10.93,3.29)   ;
			% Connection
			\draw    (265.9,99) -- (317.9,125.1) ;
			\draw [shift={(319.68,126)}, rotate = 206.66] [color={rgb, 255:red, 0; green, 0; blue, 0 }  ][line width=0.75]    (10.93,-3.29) .. controls (6.95,-1.4) and (3.31,-0.3) .. (0,0) .. controls (3.31,0.3) and (6.95,1.4) .. (10.93,3.29)   ;
			% Connection
			\draw    (170.44,161.5) -- (122.04,247.26) ;
			\draw [shift={(121.06,249)}, rotate = 299.44] [color={rgb, 255:red, 0; green, 0; blue, 0 }  ][line width=0.75]    (10.93,-3.29) .. controls (6.95,-1.4) and (3.31,-0.3) .. (0,0) .. controls (3.31,0.3) and (6.95,1.4) .. (10.93,3.29)   ;
			% Connection
			\draw    (184.56,161.5) -- (232.96,247.26) ;
			\draw [shift={(233.94,249)}, rotate = 240.56] [color={rgb, 255:red, 0; green, 0; blue, 0 }  ][line width=0.75]    (10.93,-3.29) .. controls (6.95,-1.4) and (3.31,-0.3) .. (0,0) .. controls (3.31,0.3) and (6.95,1.4) .. (10.93,3.29)   ;
			% Connection
			\draw    (366.11,172) -- (367.83,236.5) ;
			\draw [shift={(367.89,238.5)}, rotate = 268.47] [color={rgb, 255:red, 0; green, 0; blue, 0 }  ][line width=0.75]    (10.93,-3.29) .. controls (6.95,-1.4) and (3.31,-0.3) .. (0,0) .. controls (3.31,0.3) and (6.95,1.4) .. (10.93,3.29)   ;
			% Connection
			\draw    (397.9,172) -- (477.37,228.4) ;
			\draw [shift={(479,229.56)}, rotate = 215.37] [color={rgb, 255:red, 0; green, 0; blue, 0 }  ][line width=0.75]    (10.93,-3.29) .. controls (6.95,-1.4) and (3.31,-0.3) .. (0,0) .. controls (3.31,0.3) and (6.95,1.4) .. (10.93,3.29)   ;
			
			\end{tikzpicture}
			
			
			\caption{Schaubild der Aufgliederung von Substanz}
			\label{fig:1}
		\end{figure}
	\subsection{Der Körper}
		Ein K"orper ist bei Avicenna alles, wof"ur wir bis zu drei Dimensionen postulieren k"onnen. Es kann dar"uber hinaus weitere Eigenschaften haben, bspw. dass die Dimensionen begrenzt sind oder es eine Oberfl"che gibt, muss aber nicht.
	
\section{Beantwortung der Leitfragen}
\subsection{Wie beweist Avicenna die Existenz der Hyle durch Kontinuität und Trennung in den Körpern?}
\begin{itemize}
	\item K"orper k"onnen Teilung empfangen, also Form des K"orpers best"andig
	\item Die Trennung eines K"orpers trennt insbesondere auch die Kontinuit"at, welche der Ausdehnung zugrunde liegt.
	\item Somit bekommt der K"orper durch Trennung eine neue Ausdehnung, alles eigent"umliche (identit"atsstiftende?) vergeht
	\item im K"orper ist also etwas, was diese bestimmte Kontinuit"at und somit andere Eigenschaften verursacht
\end{itemize}

\subsection{Wie beweist Avicenna die Existenz der Hyle durch Aktualität und Potenzialität in den Körpern?}
\begin{itemize}
	\item K"orper hat Aktualit"at (ist nicht ein anderes Ding) oder Potentialit"at (ist nicht ein Ding)
	\item K"orper der Aktualit"at hat, hat keine Potentialit"at
	\item K"orper also zusammengesetzt aus etwas mit Aktualit"at und etwas mit Potentialit"at
	\item Form liefert Aktualit"at, Materie (Hyle) liefert Potentialit"at
\end{itemize}


\section{Quiz}
\subsection{Kann etwas sowohl Substanz als auch Akzidens sein?}
	Nein, Gegenbeispiel der Hitze im Feuer.
\subsection{Muss der Körper aus Form und Materie bestehen?}
	Nein. Er muss zwingend aus Materie/Hyle bestehen. Beispielsweise kann ein K"orper ja Potentialit"at besitzen, aber keine Aktualit"at (Beispiel Einhorn: hat eine Form, aber kein Materie)
\end{document}
