\documentclass[a4paper]{article}

\usepackage[ruled,vlined]{algorithm2e}
%opening
\title{Avicenna about corporeal matter separate from corporeal form\\\small{(The Metaphysics of the Healing, Book Two, Chapter Three, Paragraphs 2-3)}}
\author{Floris Remmert}

\begin{document}

\maketitle

%\begin{abstract}
%
%\end{abstract}
We want to show that "corporeal matter cannot exist devoid of form in actuality". We will do so using contraposition.\\
\begin{algorithm}[H]
	\caption{Assume that corporeal matter can be separated from corporeal form}
	\eIf{corporeal matter \emph{has position and a bound}}
		{\eIf{corporeal matter \emph{can be divided}}
			{corporeal matter necessarily has measure, while it was postulated not to (see Chapter 2). \emph{Contradiction.}}
			{corporeal matter is a point. A point can't exist as a single entity, separated spatially in a confine all on its own. \emph{Contradiction.}}}
		{\eIf{measure indwells in the corporeal matter \emph{all at once}}
			{\eIf{measure encountered corporeal matter in a specific confine}
				{measure encountered it while being in the confine itself. Thus, the corporeal matter was in a confine. \emph{Contradiction} with the definition of matter.}
				{no confine would have been more appropriate for corporeal matter then any other. \emph{No way to continue from here.}}}
			{If it wasn't all at once, it was a continuous process. \\
				\eIf{measure comes to corporeal matter while it's not in a confine}
				{measure connects with corporeal matter in no confine. Thus measure does not come to corporeal matter in [any] one specific confine. Thus the measure would have no confine. \emph{Contradiciton} with the definition of measure.}
				{Measure exists in every confine [corporeal matter] could possibly belong to. Thus it is not confined to some specific one. \emph{Contradiciton} with the definition of measure.}}}
\end{algorithm}
As all the possible cases lead to a contradiction, the premise must have been wrong itself.

\end{document}
